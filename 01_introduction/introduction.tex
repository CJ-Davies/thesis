\begin{quote}
	\textit{``The major challenge for the future will be effectively and cheaply to shift the sense of presence from one's own body to another, without replacing or excluding the physical world in which we all exist.''}
\end{quote}
\hfill \textit{Distributed Embodiment: Real Presence in Virtual Bodies, Waterworth \& Waterworth}
\\
\\
%=========================================================================================================
%=========================================================================================================

%\begin{itemize}
%	\item \textbf{Content -} Short (10 pages is probably far too much). High level introduction to the concepts/topics involved in the thesis, very short/broad definitions of any terms introduced in the title (eg `simultaneous presence'), overview of the rest of the document, list of contributions/donations of this document
%	\item \textbf{What has been done -} Nothing.
%	\item \textbf{What is left to do/how long should it take -} Should be one of the last things to be written, maybe a few days of writing.
%\end{itemize}

%\hrule

Talk about history of alternate realities, talk about the disappointment of VR in the 90's \& it's resurgence now thank to Oculus, etc.






Alternate realities have been a mainstay of both popular science fiction \& of serious academic research, the concept of the existence of another `there' \& how we could visit it or bring it into our `here'
keeping authors \& scientists alike fascinated for many decades.

From the `Sensorama' simulator of the 1960s to the Oculus Rift of the early 2010s

From William Gibson's `The Gernsback Continuum' \& beyond through contemporary cyberpunk literature

the `gargoyles' of MIT


The 90's saw a surge of hype over the Virtual Reality concept, however with hardware \& software not ready to meet the expectations sown among consumers by the media the bubble burst.

The advent of the smartphone lead to the dissemination of any number of Augmented Reality, an attempt to merge to some extent our real world with aspects of the virtual.


And the Media Lab invented what they would call Cross Reality that linked a real world location with a virtual other via sensors \& actuators, such that an inhabitant of one could glimpse a tantalizing insight into the goings on of the `other' although they could not see into it.

This thesis explores an alternate reality that as yet has received little attention or even a name. Parallel Reality as it came to be called builds upon the concept of Cross Reality \& its two distinct environments each `complete unto itself'. But instead of the shadow of one environment only laying upon the other by manner of sensors \& actuators, new VR technology combined with indoor positioning systems allow a user to switch between environments - to at one moment view their RW surroundings \& at the next to view the equivalent vantage in a parallel virtual environment

%=====================
%On why alternate realities are predominantly visual?

\textit{``visual ordination of intellectual knowledge''}

\textit{``Seeing remains an insistent metaphor for all of cognition only because while the ocular lobes are merely one of our brain's tentacular connections to reality, they are among the most `conscious' of their capacity for information control.''}

The Mediated Sensorium, Caroline A. Jones



Caroline A. James speaking on the work of Janet Cardiff \& George Bures Miller

\textit{``Cardiff brings segmentation into the present, crafting `sound walks' that layer an alternate reality over the fl\^aneur's perambulations - a fantastic elaboration of the kind of personal soundscape chosen by the iPod user.''}


\textit{``Sight plays such a prominent part in the mental life that the field of vision is sometimes considered almost synonymous with the field of attention.''}~\cite{Lucas1951}