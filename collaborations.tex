While the work presented in this thesis was undertaken principally by myself, it would not have been possible were it not for various collaborations. In particular the virtual reconstructions of St Andrews cathedral and St Salvator's chapel, that played a crucial role in both the Virtual Time Window and Mirrorshades projects, were created by the members of the Open Virtual Worlds research group working in collaboration with academics from the university's Art History, History and Archaeology departments, as well as with domain experts from heritage organisations including Historic Scotland and the National Trust for Scotland. Particular recognition should go to Sarah Kennedy for her critical role of modelling the reconstructions, Iain Oliver for his systems administration and for providing a Unity version of the OpenSim chapel reconstruction via a tool of his own authoring, and Richard Fawcett with the School of Art History for his invaluable input to these reconstruction processes.

%=====================

Co-authored peer reviewed papers that cover the process of creation and utilisation of these reconstructions include:

\begin{itemize}

	%iED Europe
	\item Allison, C., Campbell, A., Davies, C., Dow, L., Kennedy, S., Miller, A., Oliver, I. and Perera, I. (2012). Growing the Use of Virtual Worlds in Education: an OpenSim Perspective. Proceedings of the 2nd European Immersive Education Summit.
	
	%AINA
	\item Oliver, I., Miller, A., Allison, C., Dow, L., Campbell, A., Davies, C., and McCaffery, J. (2013). Towards the 3D Web with Open Simulator. Proceedings of the 27th IEEE International Conference on Advanced Information Networking and Applications.

\end{itemize}

%=====================

In relation to the Virtual Time Window project (see chapter \ref{chapter-vtw}), the following peer reviewed papers were produced:

\begin{itemize}

	%PGNet
	\item Davies, C., Miller, A., and Allison, C. (2012). Virtual Time Windows: Applying Cross Reality to Cultural Heritage. Proceedings of the 13th Annual Post Graduate Symposium on the Convergence of Telecommunications, Networking and Broadcasting.
	
	%iED Boston
	\item Davies, C., Allison, C., and Miller, A. (2013). PolySocial Reality for Education: Addressing the Vacancy Problem with Mobile Cross Reality. Proceedings of the 8th Immersive Education Summit.

	%Digital Heritage Marseille
	\item Davies, C., Miller, A., and Allison, C. (2013). Mobile Cross Reality for Cultural Heritage. Proceedings of the 2013 Digital Heritage International Congress (DigitalHeritage) federating the 19th Int'I VSMM, 10th Eurographics GCH, and 2nd UNESCO Memory of the World Conferences, plus special sessions from CAA, Arqueologica 2.0, Space2Place, ICOMOS ICIP and CIPA, EU projects, et al.

\end{itemize}

%=====================

The Cathedral reconstruction that the Virtual Time Window project (see chapter \ref{chapter-vtw}) made use of is covered in greater detail in:

\begin{itemize}

	%iED Europe
	\item Kennedy, S., Dow, L., Oliver, I., Sweetman, R., Miller, A., Campbell, A., Davies, C., McCaffery, J., Allison, C., Green, D., Luxford, J. and Fawcett, R. (2012). Living history with Open Virtual Worlds: Reconstructing St Andrews Cathedral as a stage for historic narrative. Proceedings of the 2nd European Immersive Education Summit.

\end{itemize}

%=====================

The design and development of the Mirrorshades platform (see chapter \ref{chapter-mirrorshades}) was presented in a poster, with accompanying abstract:

\begin{itemize}
	%VRST
	\item Davies, C., Miller, A., and Allison, C. (2014). A View from the Hill: Where Cross Reality Meets Virtual Worlds.  Proceedings of the 20th ACM Symposium on Virtual Reality Software and Technology.	
\end{itemize}

%=====================

Other work from the Open Virtual Worlds group was presented by myself on behalf of the authors:

\begin{itemize}
	
	%iED Boston
	\item Allison, C., Oliver, I., Miller, A., Davies, C. and McCaffery, J. (2013). From Metaverse to MOOC: Can the Cloud meet Scalability Challenges for Open Virtual Worlds? Proceedings of the 8th Immersive Education Summit.
	
\end{itemize}

%=====================