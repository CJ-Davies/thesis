\begin{figure}[h]
\begin{lstlisting}[language=C++, numbers=left, numberstyle=\small, stepnumber=1, frame=single, breaklines=true, backgroundcolor=\color{codebackground}, showstringspaces=false]
boost::tuple<float, float, float> LLViewerSerialMovement::latitudeLongitudeToRegionCoordinate(double newLat, double newLong, float anchorLat, float anchorLong, float scale, boost::tuple<float, float, float> anchorCoordinates) {

    double d, a, c, X, Y;

    // calculate difference in y (latitude) between anchor & new reading
    d = fabs(degreesToRadians(newLat - anchorLat));
    a = sin(d / 2) * sin(d / 2);
    c = 2 * atan2(sqrt(a), sqrt(1 - a));

    // mean radius of the Earth is 6371km (6371000m)
    d = 6371000 * c;

    // apply scale
    d *= scale;

    // sum appropriately from the anchor
    if (newLat > anchorLat) {
        Y = (anchorCoordinates.get<1>() + d);
    }
    else {
        Y = (anchorCoordinates.get<1>() - d);
    }

    // calculate difference in x (longitude) between anchor & new reading
    d = fabs(degreesToRadians((newLong - anchorLong)));
    a = sin(d / 2) * sin(d / 2) * cos(degreesToRadians(newLat)) * cos(degreesToRadians(anchorLat));
    c = 2 * atan2(sqrt(a), sqrt(1 - a));
    
    d = 6371000 * c;

    // apply scale
    d *= scale;

    // sum appropriately from anchor
    if (newLong > anchorLong) {
        X = (anchorCoordinates.get<0>() + d);
    }
    else {
        X = (anchorCoordinates.get<0>() - d);
    }

    return boost::make_tuple(X, Y, anchorCoordinates.get<2>());
}
\end{lstlisting}
\caption{Converting longitude and latitude to Second Life coordinates using haversine in modified Second Life codebase.}
\label{secondlifehaversine}
\end{figure}