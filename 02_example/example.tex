\begin{itemize}
	\item \textbf{Content -} Short (10 pages is probably far too long) example or usage scenario of how the concepts investigated in the thesis have/could be used (a `near-future usage scenario').
	\item \textbf{What has been done -} Nothing.
	\item \textbf{What is left to do/how long should it take -} Come up with a legitimate scenario, keep it short, should probably be written after the bulk of the later sections so there's a clear idea of what scenario I actually want to allude to. Perhaps a few days writing.
\end{itemize}

\hrule

\vspace{10mm}

This section presents two use cases for simultaneous presence in real and virtual environments and illustrates the two different relationships that can exist between the real and virtual environments - whether they are \textit{spatially equivalent} or not.

\section{Virtual Time Windows - Spatial Equivalence}
The ongoing Virtual Time Window (VTW) project is an application of simultaneous presence in real and virtual environments within the domain of cultural heritage that promises to further existing alternate reality work in the field by allowing simultaneous exploration of a real cultural heritage site and its virtually reconstructed counterpart via a tablet computer~\cite{Davies2012}. A visitor to a cultural heritage site, such as the ruins of the cathedral at St Andrews, uses a tablet computer that in effect presents a `window' into a virtual reconstruction of the entire site as it was at an earlier point in time, such as the cathedral in its 13th century splendour.

To maintain a natural and unhindered sense of exploration VTW does not require visitors to manually control navigation within the virtual environment. Changes in the tablet's position within the site are automatically reflected by a corresponding movement of the avatar within the virtual environment, making use of a combination of location tracking technologies. The direction that a visitor faces is monitored by magnetometer and the angle that they hold the tablet at by accelerometer; this information is reflected by the direction and pitch of the camera within the virtual environment. The resulting style of interaction is similar to using a digital camera to take a photo; the screen on the back of the camera shows what the image will look like when the shutter is released, whilst with VTW the screen on the tablet shows what the site looked like in the past.

This approach addresses the vacancy problem by presenting the user with a convenient and natural manner in which to interact with the virtual environment whilst simultaneously exploring the real environment. A tablet computer is small, light and easy to carry and by controlling the position and direction of the camera by sensing the user's physical movements the user doesn't have to pay close attention to manually navigating within the virtual environment which would risk introducing vacancy from the real environment.

By using a complete virtual environment, rather than adding sparse virtual augmentations to a user's view of the real location in an augmented reality fashion, interactions between users at the site and those who are physically elsewhere are possible via the virtual environment.

With VTW, the virtual environment is based upon the real environment. Even though certain features may differ, for example where ruins stand in the real environment a complete building may stand in the virtual environment, there is a fundamental spatial equivalence between the two environments - they are both in effect the same `location' or `place', in an abstract sense of the terms. It is this relationship that permits the project to map a user's physical position in the real environment to an equivalent position within the virtual environment, allowing them to navigate both when in affect only controlling their navigation in one.

One might consider the `Second Earth' concept to be the ultimate realisation of this scenario of spatially equivalent real and virtual environments. Discussed by Wade Roush in a 2007 article of MIT's Technology Review magazine~\cite{Roush2007}, which is cited by Lifton in his thesis, Second Earth is theorised as the combination of the notions of virtual world technology (as in Second Life) with `mirror world' technology (as in Google Earth); Second Earth theorises a virtual simulation/reconstruction of the entire physical world, such that for any location in the real world there is a corresponding location in the virtual world. Such a resource would allow for simultaneous presence in corresponding real and virtual environments to take place anywhere, rather than being restricted to specific real world locations for which a corresponding virtual location had been created, such as a cultural heritage site.

Furthermore, if one were to apply the concepts of cross reality to such a global virtual reconstruction, it would in effect create a complete parallel virtual Earth that would react in real-time to events in the real world via sensor infrastructure and would be able to affect the real world in real-time through actuator infrastructure.

Naturally the Second Earth concept will remain just that - a concept - likely for decades, as the underlying technologies and infrastructures are not yet available to us. In a blog post on the subject of virtual world/mirror world mashups~\cite{Bar-Zeev2007}, Avi Bar-Zeev estimates that the Second Life server model at the time would require 2.4 billion physical servers to host a simulation of the entire surface of the Earth, or 1400 servers just for Manhattan. In a comment on this post, Roush emphasises that his article in Technology Review was not meant to be taken as a premise for a literal Second Life/Google Earth mashup, but that they were the leading virtual world/mirror world technologies at the time and that overlap was sure to happen.

\section{Snow Crash - No Spatial Equivalence}
In the opening quote to this review, taken from Neal Stephenson's cyberpunk novel \textit{Snow Crash}, the protagonist enquires about the location of another character, called Y.T., both in the real world and in the `Metaverse'. For the sake of this discussion, this Metaverse can be considered analogous to a virtual world akin to Second Life, accessed via a head mounted display, and comprises an entirely synthetic virtual world whose locations have no counterparts in the real world. Y.T.'s response is that \textit{``In  the Metaverse, I'm on a plusbound monorail train. Just passed by Port 35.''} whilst in reality she is at a \textit{``Public terminal across the street from a Reverend Wayne's"}.

In this scenario there is no spatial equivalence between the real environment and the virtual environment - they are not the same `location' or `place' as is the case with VTW - however the concept of simultaneous presence in both can still be useful, as illustrated later in the book. Y.T. is in fact waiting for a third character, Ng, to come and collect her, which leads to the following conversation in the Metaverse between Y.T. and Ng

\begin{quote}
\textit{``\ldots you're driving?''}
\\
\\
\textit{``Yes. I'm coming to pick you up - remember?''}
\\
\\
\textit{``Do you mind?''}
\\
\\
\textit{``No,'' he sighs, as if he really does.}
\\
\\
\textit{Y.T. gets up and walks around behind his desk to look.}
\\
\\
\textit{Each of the little TV monitors is showing a different view out his van; windshield,  left  window,  right  window, rearview.  Another  one  has  an electronic map  showing his position: inbound on the San Bernardino, not far
away.}
\\
\\
\textit{``The  van  is  under  voice  command,''  he  explains.  ``I  removed  the steering-wheel-and-pedal  interface because  I found  verbal  commands  more convenient. This  is why I will  sometimes  make unfamiliar  sounds with  my voice - I am controlling the vehicle's systems.''}
\end{quote}

Ng is driving his van in the real world to come and collect the real Y.T, whilst simultaneously sitting in his virtual house in the Metaverse having a conversation with the virtual Y.T., using a series of virtual TV monitors in the Metaverse to inform him of his real surroundings and to allow him to control the van.

In this scenario there is necessarily no spatial equivalence between the real environment and the virtual environment, as the virtual environment has no spatial equivalent in the real world - the monorail and Ng's house have no real counterparts. However a lack of spatial equivalence between the real environment and the virtual environment could also arise when the virtual location does have a real counterpart, but the user isn't there. For example, in reality a user could be in London whilst in the virtual environment they could be in a location that is spatially equivalent to a real part of Hong Kong.

It is already common to see people interacting with their real location, even if that interaction is limited to just walking from A to B without bumping into too many other people or being run over by a bus, whilst simultaneously interacting with the 2D Web, social media and textual chat via mobile devices such as mobile phones and tablets. With the continued trend toward the 3D Web, it is no leap to imagine a near future in which people regularly want to interact with a 3D virtual environment that is not spatially equivalent to their current location at the same time as walking to the bus stop, thus simultaneous presence in real and virtual environments that are not spatially equivalent promises to be a desirable scenario.